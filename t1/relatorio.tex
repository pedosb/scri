\documentclass[a4paper,12pt]{article}

\usepackage[top=3cm, bottom=2cm, left=3cm, right=2cm]{geometry}
\usepackage[utf8]{inputenc}
\usepackage[portuguese]{babel}
\usepackage{booktabs}
\usepackage{multirow}
\usepackage{graphicx}
\usepackage{longtable}
\usepackage{verbatim}
\usepackage{hyperref}
\usepackage{enumitem}
\usepackage[symbol*]{footmisc}

\title{Sistemas Críticos\\[10pt]
\Large{Aplicação Tolerante a Falhas}}

\author{Pedro Batista (ext10392)\\
pedro@ufpa.br}

\begin{document}

\maketitle

\section{Implementação}

Como combinado com o professor por email, este trabalho implementará duas
variantes usando apenas dois sensores visto que será um trabalho individual. A
implementação do \textit{N-Version Programming} será representada pela variação
na linguagem de programação, para isso duas versões foram implementadas uma
linguagem \textit{Python} e outra em \textit{Java}.

\subsection{Python}\label{python}
Nessa versão optou-se por fazer um trabalho mais minucioso, onde produzimos
\textit{logs} sobre os valores de entrada e podemos analisar melhor o
comportamento do sistema. Além de conseguir tratar mais avarias que a variante
\textit{Java} deste trabalho.

\subsection{Java}\label{java}
A versão \textit{Java} foi implementada principalmente para gerar mais confiança
ao sistema, isto é, nenhuma funcionalidade adicional foi implementada, o
software atende apenas aos requisitos básicos do projeto, como será destacado ao
longo desse relatório.

\section{Definições}

\subsection{Média dos sensores}
Como nesse trabalho utilizamos apenas dois sensores como entrada, a temperatura
que será usada como entrada para o cálculo do caudal será dada por:
\begin{equation}
TS=\frac{T(i-1)*0.978+T(i-2)*1.013}{2.081}
\label{eq:media_temp}
\end{equation}
onde,
$T(i)$ é a temperatura medida no sensor $i$.

\subsection{Interpolação dos valores de entrada}\label{sec:interpolar}
Sabemos que a piscina de água da piscina não pode sofrer certas variações
bruscas em sua temperatura, isto é, podemos modelar uma função que baseada no
histórico da leitura de temperaturas nos dirá quais os valores esperados da
temperatura na piscina.

Este trabalho exige um estudo sobre as características físicas da piscina, da
água e de vários outras variáveis que fogem do escopo desse trabalho. Dessa
forma optamos por simplificar a tarefa usando um polinômio que passa por
todos os pontos anteriores para prever o ponto atual (extrapolação). Usamos
nesse trabalho a implementação de~\cite{interpolation} por SciPy.

\section{Avarias previstas}

\subsection{Nos sensores}\label{sec:av_sensor}
\begin{itemize}
	\item Definições:
		\begin{description}
			\item[Número de sensores] $N$.
			\item[Id de um sensor] $i$ dado $i\in[1..N]$.
			\item[Ciclo de leitura] leitura de todos os sensores em um instante.
		\end{description}
	\item Serão consideradas avarias nos sensores os seguintes casos:
	\begin{itemize}
		\item O formato de entrada não siga a padronização $S\_i~d.ccc$,
			onde, $d\in[0..10]$ e $c\in[0..9]$.
		\item O valor informado não pertença ao intervalo
			$[0.000,10.000]$.
		\item O valor será considerado omisso caso não esteja disponível em no máximo
			5 segundos\footnote{Implementado apenas na versão Python
			(Seção~\ref{python})\label{ft:python}}.
	\end{itemize}
	\item Os valores dos sensores são esperados na sequencia crescente do id
		dos sensores, isto é, inicializando $i=1$ lemos o sensor $i$ e esperamos
		pelo sensor $i=i+1$ até o sensor $N$ seja lido, onde o ciclo de leitura
		se completa.
		Caso isso não aconteça a Ação~\ref{ac:condicoes_iniciais} será
		executada\footref{ft:python}.
	\item Caso um sensor falhe a Ação~\ref{ac:falta_sensor} será executada.
	\item Caso todos os sensores falhem e no ciclo anterior pelo menos um deles
		não falhou, a Ação~\ref{ac:interpolar_sensores}. Se no ciclo anterior
		todos os sensores falharam (isto é, a Ação~\ref{ac:interpolar_sensores}
		será executada já foi executada) ou se este é o primeiro ciclo de
		leitura a Ação~\ref{ac:condicoes_iniciais} será
		executada~\footref{ft:python}.
	\item Caso todos os sensores falhem o sistema falhará e a
		Ação~\ref{ac:condicoes_iniciais} será executada\footnote{Implementado
		apenas na versão Java (Seção~\ref{java})\label{ft:java}}.
\end{itemize}

\subsection{Ações}
\subsubsection{Usar apenas um sensor}\label{ac:falta_sensor}
Caso um dos sensores apresente alguma avaria (Seção~\ref{sec:av_sensor}) a
temperatura a ser usada será somente a do sensor não avariado, não necessitando
dessa forma da Equação~\ref{eq:media_temp}.

\subsubsection{Retornar as condições iniciais}\label{ac:condicoes_iniciais}
Em algumas situações não poderemos confiar no histórico gerado pela
aplicação. Isso implicará na reinicialização do sistema, isto é, a próxima
entrada será considerada a primeira para o sistema, todo o histórico
guardado será apagado.
	
\subsubsection{Prever (interpolar) o valor dos
	sensores}\label{ac:interpolar_sensores}
Usamos a função definida na Seção~\ref{sec:interpolar} para prever o valor da
leitura do sensor e então assumimos que esse é verdadeiro.

\section{Conclusão}

\section{Requisitos}

\begin{itemize}
	\item Scientific algorithms library for Python (SciPy) versão 0.8.0 -
		\url{http://www.scipy.org/}
\end{itemize}

\bibliographystyle{plain}
\bibliography{bibliografia}

\end{document}
