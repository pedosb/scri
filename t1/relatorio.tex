\documentclass[a4paper,12pt]{article}

\usepackage[top=3cm, bottom=2cm, left=3cm, right=2cm]{geometry}
\usepackage[utf8]{inputenc}
\usepackage[portuguese]{babel}
\usepackage{booktabs}
\usepackage{multirow}
\usepackage{graphicx}
\usepackage{longtable}
\usepackage{verbatim}

\title{Sistemas Críticos\\[10pt]
\Large{Aplicação Tolerante a Falhas}}

\author{Pedro Batista (ext10392)\\
pedro@ufpa.br}

\begin{document}

\maketitle

\section{Definições}
Como nesse trabalho utilizamos apenas dois sensores como entrada, a temperatura
que será usada como entrada para o cálculo do caudal será dada por:
$$TS=\frac{T(i-1)*0.978+T(i-2)*1.013}{2.081}$$
onde,
$T(i)$ é a temperatura medida no sensor $i$.

Caso um dos sensores apresente alguma avaria (Seção~\ref{sec:av_sensor}) a
temperatura a ser usada será somente a do sensor não avariado, não necessitando
dessa forma da equação acima.

\section{Avarias previstas}

\subsection{Nos sensores}\label{sec:av_sensor}
\begin{itemize}
	\item Serão consideradas avarias nos sensores os seguintes casos:
	\begin{itemize}
		\item O formato de entrada não siga a padronização \verb*-S_i d.ccc-,
			onde \verb i  é um número inteiro representando o sensor correspondente,
			\verb d  é um inteiro de 0 a 10 e \verb c  é um inteiro de 0 a 9.
		\item O valor informado não pertença ao intervalo
			$[0.000,10.000]$.
		\item O valor será considerado omisso caso não esteja disponível em no máximo
			5 segundos.
	\end{itemize}
\end{itemize}

\section{Conclusão}

\end{document}
