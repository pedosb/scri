\documentclass[a4paper,10pt]{article}

\usepackage[top=3cm, bottom=2cm, left=3cm, right=2cm]{geometry}
\usepackage[utf8]{inputenc}
\usepackage[portuguese]{babel}
\usepackage{booktabs}
\usepackage{multirow}
\usepackage{graphicx}
\usepackage{longtable}
\usepackage{verbatim}
\usepackage{hyperref}
\usepackage{enumitem}
\usepackage[symbol*,stable]{footmisc}
%\usepackage[stable]{footmisc}

\title{Sistemas Críticos\\[10pt]
\Large{Aplicação Tolerante a Falhas}}

\author{Pedro Batista (ext10392)\\
pedro@ufpa.br}

\begin{document}

\maketitle

\section{Implementação}

Como combinado com o professor por email, este trabalho implementará duas
variantes usando apenas dois sensores visto que será um trabalho individual. A
implementação do \textit{N-Version Programming} será representada pela variação
na linguagem de programação, para isso duas versões foram implementadas uma
linguagem \textit{Python} e outra em \textit{Java}.

\subsection{Python}\label{python}
Nessa versão optou-se por fazer um trabalho mais minucioso, onde produzimos
\textit{logs} sobre os valores de entrada e podemos analisar melhor o
comportamento do sistema. Além de conseguir tratar mais avarias que a variante
\textit{Java} deste trabalho.

\subsection{Java}\label{java}
A versão \textit{Java} foi implementada principalmente para gerar mais confiança
ao sistema, isto é, nenhuma funcionalidade adicional foi implementada, o
software atende apenas aos requisitos básicos do projeto, como será destacado ao
longo desse relatório.

\subsection{Votador}
Como teremos que votar em um número par de variantes resolvemos fazer um votador
baseado em regras e em confiança. Dessa forma apresentamos abaixo as possíveis
saídas do votador. Segundo a especificação a saída do votador não deve
apresentar nenhuma confiança, então caso seja necessário desabilitar essa
funcionalidade basta passar a opção \verb -s  para o votador, como indicado na
Seção~\ref{sec:run_voter}.

\begin{itemize}
	\item Caso um dos programas falhem a saída será o outro.
	\item Caso a diferença entre o resultado dos dois programas seja maior que
		$\beta$ a saída escolhida será a do programa Python, pois esse trata
		mais avarias (pode ser considerado mais confiável) que o programa Java.
	\item Caso a diferença entre o resultado dos dois programas seja menor que
		$\beta$ a saída será a média dos dois resultados.
\end{itemize}

O parâmetro $\beta$ foi introduzido para o caso em que a divergência entre os
dois programas é muito grande, isso pode significar que um deles está errado.
Neste trabalho setamos $\beta=5$ empiricamente.

\section{Definições}

\subsection{Média dos sensores}
Como nesse trabalho utilizamos apenas dois sensores como entrada, a temperatura
que será usada como entrada para o cálculo do caudal será dada por:
\begin{equation}
TS=\frac{T(i-1)*0.978+T(i-2)*1.013}{2.081}
\label{eq:media_temp}
\end{equation}
onde,
$T(i)$ é a temperatura medida no sensor $i$.

\subsection{Interpolação dos valores de entrada}\label{sec:interpolar}
Sabemos que a água da piscina não pode sofrer certas variações
bruscas em sua temperatura, isto é, podemos modelar uma função que baseada no
histórico da leitura de temperaturas nos dirá quais os valores esperados da
temperatura na piscina.

Este trabalho exige um estudo sobre as características físicas da piscina, da
água e de vários outras variáveis que fogem do escopo desse trabalho. Dessa
forma optamos por simplificar a tarefa usando um polinômio que passa por
todos os pontos anteriores para prever o ponto atual (extrapolação). Usamos
nesse trabalho a implementação de~\cite{interpolation} por SciPy.

\subsection{Confiança para o programa antes do votador\footnote{Implementado
	apenas na versão Python (Seção~\ref{python})\label{ft:python}}}
	\label{sec:conf_python}
Para facilitar o trabalho de um votador, o controlador de caudal terá como saída
não só o caudal, mas também uma confiança que poderá ser posteriormente usada na
votação. Poderíamos implementar o cálculo da confiança usando modelos
matemáticos, porém nesse trabalho ela é dada por um conjunto de regras mostrado
abaixo.
\begin{description}%[label=Confiança (\arabic*)]
	\item[50] Quando o caudal calculado é resultado da
		Ação~\ref{ac:interpolar_sensores}.
	\item[70] Quando um dos sensores falhar.
	\item[90] Quando os dois sensores funcionam.
\end{description}

\subsection{Confiança do votador}
Como nesse trabalho teremos que usar um votador para um número par de entradas
decidimos implementar uma confiança de saída. Usamos para tal um conjunto de
regras mostrado abaixo, assumindo que o programa da Seção~\ref{python} fornece
uma confiança.

\begin{description}
	\item[30] Se o programa Python falhar.
	\item[70] Se a diferença do resultado dos dois programas for maior que
		$\beta$.
	\item[75] Se o programa Java falhar.
	\item[97] Se a diferença do resultado dos dois programas for menor que
		$\beta$.
\end{description}

A confiança final será dada por $0.65 * c + 0.35 * cp$ onde $c$ é a confiança
acima, e $cp$ é a confiança do programa Python apresentada na
Seção~\ref{sec:conf_python}. No caso em que o programa Python falha a confiança
será dada apenas por $c$.

\section{Avarias previstas}

\subsection{Nos sensores}\label{sec:av_sensor}
\begin{itemize}
	\item Definições:
		\begin{description}
			\item[Número de sensores] $N$.
			\item[Id de um sensor] $i$ dado $i\in[1..N]$.
			\item[Ciclo de leitura] leitura de todos os sensores em um instante.
		\end{description}
	\item Serão consideradas avarias nos sensores os seguintes casos:
	\begin{itemize}
		\item O formato de entrada não siga a padronização $S\_i~d.ccc$,
			onde, $d\in[0..10]$ e $c\in[0..9]$.
		\item O valor informado não pertença ao intervalo
			$[0.000,10.000]$.
		\item O valor será considerado omisso caso não esteja disponível em no máximo
			5 segundos\footref{ft:python}.
	\end{itemize}
	\item Os valores dos sensores são esperados na sequencia crescente do id
		dos sensores, isto é, inicializando $i=1$ lemos o sensor $i$ e esperamos
		pelo sensor $i=i+1$ até o sensor $N$ seja lido, onde o ciclo de leitura
		se completa.
		Caso isso não aconteça a Ação~\ref{ac:condicoes_iniciais} será
		executada\footref{ft:python}.
	\item Caso um sensor falhe a Ação~\ref{ac:falta_sensor} será executada.
	\item Caso todos os sensores falhem e no ciclo anterior pelo menos um deles
		não falhou, a Ação~\ref{ac:interpolar_sensores} será executada. Se no ciclo anterior
		todos os sensores falharam (isto é, a Ação~\ref{ac:interpolar_sensores}
		foi executada) ou se este é o primeiro ciclo de
		leitura a Ação~\ref{ac:condicoes_iniciais} será
		executada~\footref{ft:python}.
	\item Caso todos os sensores falhem o sistema falhará e a
		Ação~\ref{ac:condicoes_iniciais} será executada\footnote{Implementado
		apenas na versão Java (Seção~\ref{java})\label{ft:java}}.
\end{itemize}

\subsection{Ações}
\subsubsection{Usar apenas um sensor}\label{ac:falta_sensor}
Caso um dos sensores apresente alguma avaria (Seção~\ref{sec:av_sensor}) a
temperatura a ser usada será somente a do sensor não avariado, não necessitando
dessa forma da Equação~\ref{eq:media_temp}.

\subsubsection{Retornar as condições iniciais}\label{ac:condicoes_iniciais}
Em algumas situações não poderemos confiar no histórico gerado pela
aplicação. Isso implicará na reinicialização do sistema, isto é, a próxima
entrada será considerada a primeira para o sistema, todo o histórico
guardado será apagado.
	
\subsubsection{Prever (interpolar) o valor dos
	sensores}\label{ac:interpolar_sensores}
Usamos a função definida na Seção~\ref{sec:interpolar} para prever o valor da
leitura do sensor e então assumimos que esse é verdadeiro.

\section{Conclusão}
Com a implementação desse trabalho foi possível observar que a redundância
introduzida com o \textit{N-Version programming} pode melhorar substancialmente
a confiança do sistema. Observamos também a grande importância da análise
detalhada de situações de riscos, etapa essencial para o desenvolvimento do
projeto. Além disso a tomada de decisões nessas situações também tomou algum
tempo.

Neste trabalho observamos um votador um pouco diferente dos tradicionais, pois
este se baseia em regras para ilustrar a confiança que tem em sua saída.

\section{Requisitos}

O sistema foi testado em ambiente Linux, mas acredita-se que por se usar bibliotecas
cross-platform este funciona em outros sistemas operacionais.

\begin{itemize}
	\item Python 2.6 ou superior - \url{http://www.python.org}.
	\item Java Platform 6 - \url{http://www.java.com}.
	\item Scientific algorithms library for Python (SciPy) versão 0.8.0 -
		\url{http://www.scipy.org/}.
	\item Testado em Gentoo Linux - \url{http://www.gentoo.org}.
\end{itemize}

\section{Executando o aplicativo}\label{sec:run_voter}

Quatro arquivos são necessários. O do votador (\verb Voter.py ), o do aplicativo
Python (\verb Main.py ), o do aplicativo Java (\verb Main.java ) e os valores de
teste (\verb teste.txt ). Abaixo exemplificamos o uso do projeto.

Para gerar o caudal do programa Python executamos o seguinte comando:
\begin{verbatim}
$./Main.py < teste.txt > p
\end{verbatim}

O caudal do programa Java é gerado da seguinte forma:
\begin{verbatim}
$javac Main.java
$java Main < teste.txt > j
\end{verbatim}

Após ter gerado os caudais o votador pode ser executado como se segue:
\begin{verbatim}
$./Voter -j j -p p
\end{verbatim}

Opções adicionais podem ser adquiridas nos programas Python adicionando a opção
\verb -h  em sua execução. Uma opção interessante é a \verb --verbose  com a
qual podemos obter informações detalhadas sobre a execução do aplicativo.
Disponibilizamos também o script \verb run.sh  que executa os passos acima em
terminais \verb bash  da seguinte forma:
\begin{verbatim}
$./run.sh teste.txt
\end{verbatim}

\bibliographystyle{plain}
\bibliography{bibliografia}

\end{document}
