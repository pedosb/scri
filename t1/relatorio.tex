\documentclass[a4paper,12pt]{article}

\usepackage[top=3cm, bottom=2cm, left=3cm, right=2cm]{geometry}
\usepackage[utf8]{inputenc}
\usepackage[portuguese]{babel}
\usepackage{booktabs}
\usepackage{multirow}
\usepackage{graphicx}
\usepackage{longtable}
\usepackage{verbatim}
\usepackage{hyperref}

\title{Sistemas Críticos\\[10pt]
\Large{Aplicação Tolerante a Falhas}}

\author{Pedro Batista (ext10392)\\
pedro@ufpa.br}

\begin{document}

\maketitle

\section{Definições}
Como nesse trabalho utilizamos apenas dois sensores como entrada, a temperatura
que será usada como entrada para o cálculo do caudal será dada por:
$$TS=\frac{T(i-1)*0.978+T(i-2)*1.013}{2.081}$$
onde,
$T(i)$ é a temperatura medida no sensor $i$.

Caso um dos sensores apresente alguma avaria (Seção~\ref{sec:av_sensor}) a
temperatura a ser usada será somente a do sensor não avariado, não necessitando
dessa forma da equação acima.

\section{Avarias previstas}

\subsection{Nos sensores}\label{sec:av_sensor}
\begin{itemize}
	\item Definições:
		\begin{itemize}
			\item $i$ é o id de um sensor.
			\item $N$ é o número de sensores.
			\item $i\in[1..N]$.
		\end{itemize}
	\item Serão consideradas avarias nos sensores os seguintes casos:
	\begin{itemize}
		\item O formato de entrada não siga a padronização $Si~d.ccc$,
			onde, $d\in[0..10]$ e $c\in[0..9]$.
		\item O valor informado não pertença ao intervalo
			$[0.000,10.000]$.
		\item O valor será considerado omisso caso não esteja disponível em no máximo
			5 segundos.
	\end{itemize}
	\item Os valores dos sensores são esperados na sequencia crescente do id
		dos sensores, isto é, inicializando $i=1$ lemos o sensor $i$ e esperamos
		pelo sensor $i=i+1$ até o sensor $N$ seja lido, onde o ciclo de
		leitura se completa e voltamos a esperar o sensor $1$. Caso isso não aconteça a
		Ação~\ref{ac:condicoes_iniciais} será executada.
\end{itemize}

\section{Conclusão}

\section{Requisitos}

\begin{itemize}
	\item Scientific algorithms library for Python (SciPy) versão 0.8.0 -
		\url{http://www.scipy.org/}
\end{itemize}

\end{document}
